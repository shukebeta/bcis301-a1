%!TEX root = ../assignment1.tex

\section{Case Study}

A case study \citetitle{case_study} is given to compare with our findings in previous section. The case study investigates how to succeed in implementation of ICT risk management. The authors\parencite{case_study} make a survey in 3 Thai companies: a bank, a telecommunications company and a software development company. The case study unfolds with the following sections.

\subsection{Introduction, Terms And Models}

In introduction, the authors describe what ICT risk management is and its background. Furthermore, two approaches are introduced. The first one is called "ICT governance", the top-down strategy. This strategy is from Control Objectives for the Information and related Technology (COBIT) framework, which concerns more with a business view than with technical solutions to ICT risk management. Conversely, the other one is the bottom-up strategy, "IS governance". ISO/IEC 17799 standard is a good example for the bottom-up strategy, which focuses on detailed technical solution to handle risk in ICT projects. However, some researchers argue the ISO/IEC 17799 is biased, due to its one-sided focus on technical implementation. Eventually, the author throw two questions that the case study originates: 1) what is the ICT risk management like in organizations 2) how do organizations practice ICT risk management given the two strategies.

To extend the topic, the authors suggest internal control and audit are the main roles in risk management. Multiple sources are referenced to show the importance and the roles in organizations. An ICT governance standard is introduced to list detailed sub-concepts. Similarly, the significance of IS governance is introduced. Several sources support this idea and provide a list of benefits.

With joint effort of ICT governance and IS governance, effective corporate governance can be achieved, as the authors argue. An hierarchy model of administration is illustrated to clarify the joint effort.

Given the basic idea of ICT governance and IS governance, the authors further the research by giving a comprehensive review on the COBIT framework and the ISO/IEC 17799 framework in terms of the strategies when coping with the ICT risk management. These two frameworks are also known as the "top-down" and "bottom-up" approaches.

To summarize, ICT risk management can be implement using ICT governance strategy and IS governance strategy as what the author argue. Two frameworks, namely the COBIT framework and the ISO/IEC 17799 framework, are used as approaches in the respective strategies mentioned above. This is the theoretical basis of methodology.

\subsection{Methodology In Case Study}

An interview is designed to examine the usage of the aforementioned frameworks, which indirectly, investigates how the interviewee company runs ICT risk management, given the theoretical basis illustrated in the previous subsection. 3 Thai companies that use both technological and accounting tools are chosen to ensure the prerequisites is met. The authors design questions in a way that covers 19 issues in 4 subareas of ICT risk management: structure, process, control and strategies in organization, with open-ended questions. A comparison is given among the 3 companies(a bank, a telecommunications company and a software development company).

\subsection{Findings Of Case Study}

The authors summarize the 4 key areas in application ICT risk management. Organizational Structure, organizational process, organizational control and organizational ICT strategy. The first 3 areas are compared while the fourth one is a recommendation list.

\subsubsection{Organizational Structure}
In terms of organizational structure, the authors compare 3 aspects of differences. Firstly, risk management is governed by a separate committee in company A while company B and C is not. The standalone risk management committee is considered when more setting of responsibility is given. Secondly, the position level of ICT risk management treatment differentiates. Both company A and company B practice ICT risk management in corporate level along with operational level. Company C, however, does the management embedded in application software. The authors believe the difference in business causes different structures regarding ICT risk management. Lastly, both company A and B have set the term of ICT activities within ICT policy and information security activities as in IS policy. On the contrary, company C focuses on IS(information system) only. The heavy involvement with software development is deemed as the cause of company C being different.

\subsubsection{Organizational Process}
Risk management practice can be at corporate level and operational level. Company A uses top-down and bottom-up approaches for the corporate level and the operational level receptively. Company also covers both levels but mainly uses top-down approach at corporate level. However, Company C mainly focuses on operational level for the IS system only.

\subsubsection{Organizational Control}
Organization A, B and C concentrate on the control of business, service, ICT and IS functions simultaneously, but in two different patterns. In company A and B, all areas of business, service, ICT and IS functions are monitored concurrently. Company C focusing on separate projects, controls only 3 areas of business, ICT and IS, ignoring the service function.

\subsubsection{Organizational ICT strategy}
Instead of comparison, a list of recommendation is provided for this area in ICT risk management. 1) a company should consider risk planning at the corporate level within the overall ICT plan. 2) a company should consider leading to the operational plan by covering both ICT management and IS management. 3) ICT risk management should be considered at both corporate and operational levels as ICT risk management and ICT project risk management respectively. 4) corporate level plan of ICT risk management should be planned first and drive the operational-level ICT project risk management plan. 5) successful ICT risk management planning should focus the collaboration between the management level activities and the operational level activities.

% todo 把对比的表格搬过来凑点字数?