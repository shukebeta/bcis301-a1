%!TEX root = ../assignment1.tex

\section{Methodology}
% 总流程
For this study, a qualitative analysis is first carried out, gathering all the possible solutions that the researchers suggest regarding success, failure and risks identified in project management, change management and risk management. All the 6 sources are in the reference list at end of the this article. They are from professional journals and forums published within 10 years from now. which reliability and professionalism can be assured.

% 列举文章和主题
In our selection of literature, \citetitle{1} and \citetitle{2} are on project management. \citetitle{3} is on change management while \citetitle{4} and \citetitle{5} are on risk management. \citetitle{6} gives an overview and an update of information system success and failure, which is composed by an affiliation of famous scholars in the field.

% 如何 coding
When utilizing the aforementioned selection of literature for qualitative analysis, we seek to identify issues, solutions and/or conclusions that authors have deduced or based on. Furthermore, the phases and the roles of stakeholders involved are also noted when a possible clues is identified. With the identified data, we reword the statements as possible proposals. All possible proposals with notation are what we have captured for subsequent evaluation process.

% 如何做 evaluation framework
With this captured data, an evaluation is processed where all the recommendation is listed and evaluated by how a given proposal affects on the phases of a project and on the roles of stakeholders in corresponding phases. To better employ the prioritized lists of proposals by some researchers who use statistical models, a cross-contextual coefficient is introduced during the evaluation process. In the evaluation process, global influence score of a proposal in a specific research context is calculated by the following formula \ref{brief}.
\begin{equation}
g_{C_{i},j} = \mathit{k_{C_{i},j}}l_{C_{i},j},\ C_{i,j} \in C_{i},\ C_{i} \in \mathbb{C}
\label{brief}
\end{equation}

In formula \ref{brief}, $\mathbb{C}$ denotes the set of all research contexts. And for each research context $C_{i}$ which is an element in $\mathbb{C}$, $C_{i,j}$ denotes a proposal in the given research context $C_{i}$. $g_{Ci,j}$ is the global influence score of a proposal in a given research context $C_{i}$. Similarly, $l_{C_{i},j}$ is the local influence score of the same proposal; $\mathit{k_{C_{i},j}}$ is the cross-contextual coefficient for the same proposal. Each element on the right hand side of the formula can be calculated respectively as follows.
\begin{equation}
\mathit{k_{C_{i},j}} = \frac{f_{C_{i,j}}}{\bar{f_{C_i}}}
\label{coefficient}
\end{equation}
\begin{equation}
\bar{f_{C_i}} = \frac{1}{n}\left (\sum_{m=1}^n{f_{C_{i,m}}}\right)
\label{mean}
\end{equation}
\begin{equation}
l_{C_{i},j} = |P_{C_{i,j}}||R_{C_{i,j}}|
\label{local_influence}
\end{equation}

In formula \ref{coefficient} and \ref{mean}, $f_{C_{i,j}}$ denotes the local influence score in a research context $C_{i}$ and $\bar{f_{C_i}}$ is the arithmetic mean value of all local influence scores of proposals in a given context $C_{i}$.

In formula \ref{local_influence}, $l_{C_{i},j}$ is the local influence score of a given proposal $C_{i,j}$. $|X|$ denotes the cardinality of the given set $X$. $P_{C_{i,j}}$ is the set of affected phases by the proposal $C_{i,j}$. Similarly, $R_{C_{i,j}}$ is that of affected roles of stakeholders in the same context.

Thus, the global influence score of a given proposal that denotes as $g_{C_{i},j}$ can be calculated as follows:
\begin{equation}
g_{C_{i},j} =
\frac
% 分子 k local=(P R)
{f_{C_{i,j}} |P_{C_{i,j}}| |R_{C_{i,j}}|} 
% ----分式线-----
% 分母 f平均
{\frac{1}{n}\left (\sum_{m=1}^n{f_{C_{i,m}}}\right)}
% 满足条件
,\ 
C_{i,j} \in C_{i},\ C_{i} \in \mathbb{C}
\label{final}
\end{equation}

An higher score of $g_{C_{i},j}$ indicates that the proposal has a wider and deeper influence on project success. We post-process the result of score by descending ranking, in which we select the ones higher than average value. This subset of proposals are tagged by sub-topics and enrolled in a checklist as solutions.
